The detection of ideological bias in news has been addressed through various lenses, ranging from structural networks to advanced linguistic modeling. 
Early systematic studies, such as those by Efron [2004], focused on structural elements like co-citation information to estimate political orientation. 
Similarly, Gentzkow and Shapiro [2006] proposed economic models to analyze how media reputations influence audience alignment. Advancing this structural 
view, Lin et al. [2011] combined social network analysis with NLP to quantify bias through interaction graphs between sources and entities. While effective, 
these methods often struggle with documents containing sparse hyperlink data or those isolated from established citation networks.

The analysis of textual content remains the most direct approach to bias detection. Dallmann et al. [2015] examined framing in German newspapers, 
identifying how word choice and theme frequency reflect ideological tendencies. Gangula et al. [2019] utilized attention mechanisms to capture bias specifically 
in headlines, highlighting their role in shaping reader perception. In a broader context, Baly et al. [2020] developed a robust framework using BERT and LSTM 
for textual classification. Their findings indicated that while high accuracy could be achieved on known sources, these models often failed to generalize when 
faced with news articles from unseen sources, a limitation that remains a central challenge in the field.

Social media platforms offer alternative metadata for bias inference through user interactions and demographic data. Rao and Spasojevic [2016] utilized word 
embeddings and LSTM networks to classify the political leaning of tweets, achieving high accuracy in binary U.S. political contexts. Elejalde et al. [2017] 
automated political orientation tests by analyzing news outlets' tweets and vocabulary. Furthermore, Ribeiro et al. [2018] inferred ideological orientation 
by analyzing ad interface data on Facebook and Twitter, estimating bias based on the demographic profile of a source's audience. Despite their effectiveness, 
these methods remain dependent on external platform data and are often restricted to highly polarized, country-specific scenarios.

The emergence of Large Language Models (LLMs) has introduced new paradigms, such as the \textit{POLITICS} framework [Liu et al., 2022], which uses ideology-driven 
pre-training to compare articles across different media outlets. Similarly, \textit{IndiVec} [Lin et al., 2024] leverages the instruction-following capabilities of 
LLMs combined with vector databases to provide fine-grained bias indicators. However, the use of LLMs introduces complexities, as studies by Rozado [2023] show 
that architectures like ChatGPT manifest consistent preferences for specific political viewpoints. Furthermore, significant disparities have been observed between 
LLM predictions and human perception of bias [Lin et al., 2025], necessitating more transparent and specialized architectures.

To address these gaps, this research explores Deep Metric Learning to create structured feature spaces where ideological proximity is modeled by distance metrics. 
This study introduces a source-independent method focusing on the article’s raw text, utilizing localized textual segments to extract source-agnostic signatures. 
By applying \textit{Triplet/Contrastive Losses}, the approach clusters articles by ideological similarity more cohesively than traditional classification heads, capturing 
the core editorial framing while maintaining the efficiency of standard Transformer encoders. This shift to metric representation learning extracts source-agnostic signatures, 
overcoming generalization hurdles and providing a computationally sustainable alternative to LLMs without compromising detection accuracy.


\subsection{Summary and Comparison}

The methods reviewed indicate that most approaches are either limited to specific case studies, restricted to polarized binary classifications, or dependent on external metadata. 
Structural methods often fail with sparse data, while social media-based methods remain platform-dependent. Although LLM approaches are powerful, they introduce concerns regarding 
inherent model bias and high computational costs. Table~\ref{tab:sum_rela} summarizes these works, contrasting traditional models with recent LLM-based state-of-the-art results to 
highlight the critical trade-off between computational cost and predictive performance.

\begin{table*}[!ht]
\centering
\caption{Summary of the related work including LLM-based approaches and datasets.} 
\label{tab:sum_rela}
\resizebox{\textwidth}{!}{%
    \begin{tabular}{l l l l l}
    \toprule
    \textbf{Author} & \textbf{Strategy} & \textbf{Dataset} & \textbf{Bias Classes} & \textbf{Performance} \\
    \midrule
    Efron [2004] & Co-citation/hyperlinks & Open Directory & Liberal/Conservative & 77.5\% (Accuracy) \\
    Lin et al. [2011] & Social Network Analysis & OpenCongress & Democrat/Republican & - \\
    Dallmann et al. [2015] & Similarity metrics & German federal parliament & German political party & - \\
    Rao et al. [2016] & Word embeddings + LSTM & Twitter API & Democrat/Republican & 87\% (Accuracy) \\
    Elejalde et al. [2017] & Rank difference & Twitter API & Liberal/Conservative & - \\
    Ribeiro et al. [2018] & Audience statistics & Facebook API & Liberal/Conservative & - \\
    Baly et al. [2020] & BERT + LSTM & ABP & Left/Center/Right & 79.83\% (Macro F1) \\
    Liu et al. [2022] (POLITICS) & Ideology-driven Pre-training & BIGNEWS  & Left/Center/Right & 82.2\% (Avg. Acc.)  \\
    Lin et al. [2024] (IndiVec) & LLM + Vector Database & FLIPBIAS  & Left/Center/Right & 81.3\% (Accuracy)  \\
    Lin et al. [2025] & LLM (zero-shot) & FLIPBIAS & Left/Center/Right & 39.4\% (Macro F1) \\
    \bottomrule
    \end{tabular}%
}
\end{table*}


In this context, the model by Baly et al. [2020] is adopted as the primary baseline for direct comparison, as it was specifically trained and evaluated on the Articles Bias Prediction (ABP) dataset. Although LLM-based frameworks like \textit{POLITICS} 
report high metrics ($82.2\%$ average accuracy) using massive pre-training on the BIGNEWS dataset, Baly et al. [2020] provides more comparable experimental 
conditions to evaluate the impact of metric learning architectures. This choice allows for isolating the model's effectiveness in identifying ideological 
nuances without relying on massive scales or external knowledge bases.