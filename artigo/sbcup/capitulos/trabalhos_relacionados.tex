A literatura aborda a detecção de viés ideológico sob diversas óticas, desde redes estruturais \cite{efron2004liberal} até modelos econômicos de reputação midiática \cite{gentzkow2006media}. 
Avanços integraram análise de redes sociais e PLN para quantificar o viés via gráficos de interação \cite{lin2011more}. Contudo, tais métodos enfrentam limitações severas em documentos com 
dados esparsos de hiperlinks ou isolados de redes de citação estabelecidas, dificultando a generalização pretendida.

A análise textual direta foca na escolha lexical e frequência temática como reflexos de tendências ideológicas \cite{dallmann2015media}, utilizando inclusive mecanismos de atenção para capturar o enquadramento (\textit{framing}) 
em manchetes \cite{gangula-etal-2019-detecting}. Estruturas baseadas em BERT e LSTM \cite{baly2020we} alcançam alta precisão em fontes conhecidas, mas demonstram fragilidade ao processar artigos de domínios novos. 
Esse desafio permanece central na área, exigindo modelos que identifiquem assinaturas ideológicas independentemente da identidade da fonte.

Alternativamente, metadados de redes sociais, como interações no Twitter [Rao and Spasojevic, 2016; Elejalde et al., 2017] e perfis demográficos de audiência no Facebook [Ribeiro et al., 2018], 
auxiliam na inferência de inclinação política. Entretanto, essas abordagens são dependentes de APIs de terceiros e restringem-se frequentemente a cenários de alta polarização binária.
Tal dependência reforça a necessidade de soluções que operem exclusivamente sobre o conteúdo textual bruto.

Recentemente, o surgimento de LLMs introduziu paradigmas como o framework \textit{POLITICS} \cite{liu-etal-2022-politics} e o sistema \textit{IndiVec} \cite{lin2024indivec}. Apesar do alto desempenho, 
esses modelos manifestam preferências políticas consistentes em suas arquiteturas e falham em alinhar suas predições com a percepção humana em tarefas de zero-shot \cite{lin-etal-2025-investigating}. 
Este trabalho propõe o uso do \textit{Deep Metric Learning} como uma alternativa robusta e transparente, superando as limitações de generalização de \textit{baselines} estabelecidos \cite{baly2020we} 
sem a complexidade computacional excessiva dos modelos de larga escala.
